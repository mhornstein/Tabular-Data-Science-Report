\documentclass[11pt]{article}
\pagestyle{empty}
\usepackage{enumerate}

\begin{document}

Video 1 (intro):

hello!
This is my first latex file.

This is a new pharagraph.\\this is a new line but in the same paragraph.

Let start math mode: $(x+1)$ (this is inline math mode).
Not that this is display math mode: $$(x+1)$$

Video 2 (math):

Look at the difference: $x^23$ vs. $x^{23}$. The reason: only the first digit goes to the exponent.

Another example: $x^{x+y^2+2}$

The same happens with subscript: $x_23$ vs. $x_{23}$.

Here we can have several subscript: $x_{1_{2_3}}$

dots can be done like that: $x_1 \ldots x_{100}$.

ldots = dots in the lower space. in the center - use cdots: $x_1 \cdots x_{100}$.

Let's move to Greek letters:

$\pi  vs  \Pi, \alpha$
$$ A = \pi r^2$$

note the difference between $\sin x$ and simple $sinx$.
Another trick for example: $y=\cos \theta$

Lets see logs:

$y =log x$ vs $y=\log x$.

we can do it as well: $y = \log_5 x$ and $y_4 = \ln x$.

for root: 
$\sqrt{64}, \sqrt[3]{64}$.

We can create fractions:

$\frac{2}{3}$

another option is: $\displaystyle \frac{2}{3}$ vs $\frac{2}{3}$

we

can

have line breaks like \\[16pt] that

Video 3: bullets:


number list called enumarate list:

\begin{enumerate}
\item pencils
\item pens
\item notebook
	\begin{enumerate}
	\item notes
	\item assessments
		\begin{enumerate}
		\item first item
		\item second item
		
		\end{enumerate}
	
	\end{enumerate}
\end{enumerate}

Bullets: we use itemize:
\begin{itemize}

\item pencils
\item pens
\item notebook

\end{itemize}

some more options for enumrating:
\begin{enumerate}[A.]
\item first Item
\item second Item

\end{enumerate}

And another one:
\begin{enumerate}[i.]
\item first Item
\item second Item

\end{enumerate}

\pagebreak

And another that starts from 6:

\begin{enumerate} \setcounter{enumi}{5}
\item first Item
\item second Item
\item third Item

\end{enumerate}

And another!
\begin{enumerate}
\item[] first Item
\item[] second Item
\item[] third Item

\end{enumerate}

And Last!
\begin{enumerate}
\item[ONE] first Item
\item[TWO] second Item
\item[THIRD] third Item

\end{enumerate}



\end{document}